\newcommand{\MyTitle}{Cricket Scenario}
\documentclass[a5paper,10pt]{article}

\usepackage[utf8]{inputenc}
\usepackage[swedish]{babel}
\usepackage[T1]{fontenc}
\usepackage[pdftex]{graphicx}
\usepackage{calc,xparse,geometry}
\usepackage{wrapfig}
\graphicspath{ {./../Bilder/} }
\author{Gustav Ruthgård}
\title{\MyTitle}
\date{\today}

\usepackage{xkeyval}

\makeatletter
\define@cmdkey{rpg}[char@]{name}[]{}
\define@cmdkey{rpg}[char@]{class}[]{}
\define@cmdkey{rpg}[char@]{kropp}[]{}
\define@cmdkey{rpg}[char@]{reaktion}[]{}
\define@cmdkey{rpg}[char@]{sinne}[]{}
\define@cmdkey{rpg}[char@]{special}[]{}
\define@cmdkey{rpg}[char@]{width}[\linewidth]{}

\define@cmdkey{move}[char@]{rubrik}[]{}
\define@cmdkey{move}[char@]{beskrivning}[]{}
\define@cmdkey{move}[char@]{grundegenskap}[]{}
\define@cmdkey{move}[char@]{lyckat}[]{}
\define@cmdkey{move}[char@]{mitten}[]{}
\define@cmdkey{move}[char@]{misslyckat}[]{}
\define@cmdkey{move}[char@]{val}[]{}
\define@cmdkey{move}[char@]{width}[\linewidth]{}

\define@cmdkey{scenen}[char@]{typ}[]{}
\define@cmdkey{scenen}[char@]{namn}[]{}
\define@cmdkey{scenen}[char@]{rubrik}[]{}
\define@cmdkey{scenen}[char@]{plats}[]{}
\define@cmdkey{scenen}[char@]{birollett}[]{}
\define@cmdkey{scenen}[char@]{birolltva}[]{}
\define@cmdkey{scenen}[char@]{birolltre}[]{}
\define@cmdkey{scenen}[char@]{width}[\linewidth]{}

\newcommand{\statlabel}{\textsc}

\usepackage{calc,xparse}
\newlength\mycardwidth
\setlength\mycardwidth{.75\textwidth}
\NewDocumentCommand \boxme { O{} }{%
  \fbox{%
    \parbox{\linewidth-2\fboxrule-2.5\fboxsep}{\strut #1}%
  }%
}
\NewDocumentCommand \pageme { O{} O{} }{%
  \begin{minipage}{#1\mycardwidth}
    \centering\boxme[#2]
  \end{minipage}%
}
\NewDocumentCommand \cardme { O{.75\textwidth} +m }{%
  \setlength\mycardwidth{#1-2\fboxrule-2\fboxsep}%
  \fbox{%
    \begin{minipage}{\mycardwidth}
      \centering
      #2%
    \end{minipage}%
  }%
}

\newcommand{\setcharacterstats}[1]{%
  \begingroup
  \setkeys{rpg}{name,class,kropp,reaktion,sinne,special,width,#1}%
  \noindent
  \cardme[\textwidth]{%
    \pageme[.69][\statlabel{Namn}: \char@name]%
    \pageme[.31][\statlabel{\char@class}]

    \pageme[.33][\statlabel{Kropp}: \char@kropp]%
    \pageme[.34][\statlabel{Reaktion}: \char@reaktion]%
    \pageme[.33][\statlabel{Sinne}: \char@sinne]
    \raggedright \textit{Sätt ut +2, +1, 0}

    \pageme[][\statlabel{Specialdrag}:]
    \raggedright \textit{\char@special}
  }
}
\newcommand{\displaymove}[1]{%
  \begingroup
  \setkeys{move}{rubrik,beskrivning,grundegenskap,lyckat,mitten,misslyckat,val,width,#1}%
  \noindent
  \cardme[\textwidth]{%
    \pageme[.69][\statlabel{\char@rubrik}]%
    \pageme[.31][\statlabel{Slå} +\char@grundegenskap]

    \pageme[.99][\char@beskrivning]

    \pageme[.31][\statlabel{10+}]%
    \pageme[.69][\char@lyckat]

    \pageme[.31][\statlabel{7-9}]%
    \pageme[.69][\char@mitten]

    \pageme[.31][\statlabel{2-6}]%
    \pageme[.69][\char@misslyckat]

    \pageme[.99][\char@val]%
  }
}
\newcommand{\satscen}[1]{%
  \begingroup
  \setkeys{scenen}{typ,namn,rubrik,plats,birollett,birolltva,birolltre,width,#1}%
  \noindent
  \cardme[\textwidth]{%
    \pageme[1][\statlabel{\textbf{\char@typ}}]

    \pageme[1][\statlabel{Rubrik:} \char@rubrik]

    \pageme[1][\statlabel{Plats:} \char@plats]

    \pageme[1][\statlabel{Huvudroll:} \char@namn]

    \pageme[1][\statlabel{Biroller: } \char@birollett, \char@birolltva]

    \pageme[1][\statlabel{Biroller: } \char@birolltre]
  }
}
\newcommand{\rollformular}[1]{%
  \begingroup
  \setkeys{rpg}{name,class,kropp,reaktion,sinne,special,width,#1}%
  \noindent
  \cardme[\textwidth]{%
    \pageme[.69][\statlabel{Namn}: \char@name]%
    \pageme[.31][\statlabel{Koncept}:]
    \pageme[][\statlabel{Rädsla}:]
    \pageme[][\statlabel{Drivkraft}:]

    \pageme[.33][\statlabel{Kropp}: \char@kropp]%
    \pageme[.34][\statlabel{Reaktion}: \char@reaktion]%
    \pageme[.33][\statlabel{Sinne}: \char@sinne]
    \pageme[][\statlabel{Specialdrag}:]
  }
}

\define@cmdkey{bild}[bild@]{namn}[]{}
\define@cmdkey{bild}[bild@]{sida}[r]{}
\newcommand{\bild}[1]{%
  \begingroup
  \setkeys{bild}{namn,#1}%
  \begin{wrapfigure}{\bild@sida}{0.25\textwidth} %this figure will be at the right
      \centering
      \includegraphics[width=0.25\textwidth]{\bild@namn}
  \end{wrapfigure}
}
\makeatletter

\begin{document}
\maketitle
%\clearpage
%\clearpage
\part{Spelardrag}
Din rollperson kan utföra alla dessa drag som en följd av att du agerar på olika sätt i en scen. När du säger vad din rollperson gör kan SL då be dig slå för ett av dessa drag.
\section{Attackera}
\textit{När du attackerar någon som kämpar tillbaka välj hur}

och slår \textbf{+Kropp}
\begin{itemize}
  \item[10+] Du lyckas och undviker motståndarens attack.
  \item[7-9] Du attackerar motståndaren men till ett pris. Spelledaren väljer 1:
  \begin{itemize}
    \item Du träffas av ett motanfall.
    \item Du gör mindre skada.
    \item Du förlorar något.
    \item Du gör av med all ammo.
    \item Du utsätts för ett nytt hot.
    \item Du får senare problem.
  \end{itemize}
  \item[2-6] SL gör ett mjukt eller hårt drag.
\end{itemize}
\clearpage
\section{Undvika}
\textit{När du undviker, blockerar eller parerar skada} slå \textbf{+Kropp}.
\begin{itemize}
  \item[10+] Du klarar dig helt oskadd.
  \item[7-9] Du klarar dig undan det värsta av skadan men spelledaren väljer om du hamnar i ett dåligt läge, förlorar något eller om du inte helt undviker skada.
  \item[2-6] Du reagerar för långsamt eller gör en felbedömning: kanske undviker du inte alls eller så hamnar du i ett värre läge än du började i. SL gör ett drag.
\end{itemize}
\section{Tar skada}
\textit{När du utsätts för skada} slå \textbf{+Kropp}. Har du skydd lägger du till värdet till slaget.
\begin{itemize}
  \item[10+] Du biter ihop och kan fortsätta som vanligt.
  \item[7-9] Du står fortfarande på benen men spelledaren väljer:
  \begin{itemize}
    \item Skadan får dig att hamna ur balans
    \item Du tappar något
    \item Du får ett \textbf{allvarligt sår}
  \end{itemize}
  \item[2-6] Skadan är överväldigande. Du väljer om du:
  \begin{itemize}
    \item Är utslagen för resten av scenen(SL avgör om du även får ett \textbf{allvarligt sår}).
    \item Får ett \textbf{kritisk sår} men är vid medvetande (har du redan ett kritiskt sår kan du inte välja detta alternativ igen).
    \item Dör.
  \end{itemize}
\end{itemize}
\subsection{Allvarligt sår}
Såret behöver någon typ av vård eller tid för att läka men blir inte värre av sig självt. Alkohol och smärtstillande droger kan ta bort avdraget som såret ger om så bara tillfälligt.
\subsection{Kritiskt sår}
Såret kommer inte läka av sig själv utan förvärras. Den som är kritisk sårad måste ha vård inom kort för att inte dö.
\subsection{Avdrag från sår}
Om du har icke-stabiliserade allvarliga och/eller kritiska sår drabbas du av avdrag, enligt nedan.
Om du har
\begin{itemize}
  \item ...minst ett allvarligt sår; -1 Kropp
  \item ...ett kritiskt sår; -1 Kropp
  \item ...både minst ett allvarlig sår och ett kritiskt sår; -2 Kropp
\end{itemize}
Om du dricker alkohol, tar smärtstillande, eller bedövar din smärta på liknande sätt, neutraliserar du avdraget från dina \textbf{allvarliga sår} under en kortare tidsperiod, vanligen under en scen. Detta gäller inte kritiska sår.
\clearpage
\section{Överblicka}
\textit{När du överblickar situationen} slå \textbf{+Reaktion}. Vid lyckat kan du ställa frågor till spelledaren. När du agerar på spelledarens råd ta +1 på ditt slag.
\begin{itemize}
  \item[10+] Ställ 2 frågor.
  \item[7-9] Ställ 1 fråga.
  \item[2-6] Du får ställa en fråga ändå men du drar till dig oönskad uppmärksamhet eller utsätter dig för fara.
\end{itemize}
Frågor
\begin{itemize}
  \item Vad är min bästa väg förbi hindret?
  \item Vad är det största hotet mot mig?
  \item Vad kan jag använda till min fördel?
  \item Vad behöver jag vara vaksam på?
  \item Finns det något gömt här?
  \item Är det något som är underligt?
\end{itemize}
\section{Läsa av en person}
\textit{När du läser av en person} slå \textbf{+Reaktion}
\begin{itemize}
  \item[10+] Ställ 2 frågor.
  \item[7-9] Ställ 1 fråga.
  \item[2-6] Du får ställa en fråga ändå men du drar till dig oönskad uppmärksamhet eller utsätter dig för fara.
\end{itemize}
Medan du talar med personen du läser av kan du spendera dina frågor 1 för 1 för att ställa deras spelare/spelledaren frågor:
\begin{itemize}
  \item Ljuger du?
  \item Vad känner du just nu?
  \item Vad tänker du göra?
  \item Vad önskar du att jag gör?
  \item Hur kan jag få dig att ... ?
  \item Är det något som är underligt?
\end{itemize}
\section{Undersöka}
\textit{När du undersöker någonting}, slå \textbf{+Sinne}. Om du lyckas så finner du alla direkta ledtrådar och får ställa frågor för att få mera information.
\begin{itemize}
  \item[10+] Fråga två frågor
  \item[7-9] Fråga en fråga, men svaret kostar dig något. SL bestämmer vad; du behöver någon eller något för att förstå svaret, eller det tar dig extra tid att få reda på svaret.
  \item[2-6] Du får fråga en fråga ändå, men du utsätts för oväntad fara eller annan kostnad.
\end{itemize}
Frågor:
\begin{itemize}
 \item Hur kan jag få reda på mer om vad jag undersöker?
 \item Vad säger min magkänsla om vad jag undersöker?
 \item Är det något konstigt med det jag undersöker?
\end{itemize}
\section{Övertala}
\textit{När du övertalar en spelledarperson genom förhandling, argumentation eller utifrån en maktposition} slå \textbf{+Sinne}.
\begin{itemize}
  \item[10+] Hon ger vika.
  \item[7-9] Hon gör som du vill men (SL väljer):
  \begin{itemize}
    \item Hon är inte nöjd och ber om mer i gengäld.
    \item Övertalningen skapar problem vid ett senare tillfälle.
    \item Hon ger med sig men är osäker (övertalningens effekt är bara tillfällig).
  \end{itemize}
  \item[2-6] Övertalningsförsöket har oavsiktliga konsekvenser. SL gör ett drag.
\end{itemize}

\textit{När du försöker övertala en rollperson} slå \textbf{+Sinne}.
\begin{itemize}
  \item[10+] Rollpersonen väljer själv om hon ger med sig eller inte, du ger dock båda effekterna nedan.
  \item[7-9] Rollpersonen väljer själv om hon ger med sig eller inte, du väljer dock en effekt nedan.
  \item[2-6] Rollpersonen är fri att göra som hon vill och har +1 på nästa slag mot dig. Ingen av effekterna nedan gäller.
\end{itemize}
Effekter:
\begin{itemize}
  \item Hon motiveras att göra som du vill (hon får +1 på nästa slag).
  \item Hon drabbas av tvivel om hon inte gör som du vill (hon får -1 Välmående).
\end{itemize}
\section{Agera under hot}
\textit{När du gör något riskabelt, är under tidspress eller försöker undkomma fara, berättar SL vad hotet är}, slå \textbf{+Sinne} för att agera trotts hotet.
\begin{itemize}
  \item[10+] Du gör det.
  \item[7-9] Du gör det, men tvekar, blir fördröjd, eller måste reagera på en komplikation - SL ger dig ett oväntat resultat, ett högt pris eller ett svårt val.
  \item[2-6] Det blir konsekvenser, du gör misstag eller så utsätts du för fara. SL gör ett mjukt eller hårt drag.
\end{itemize}
\section{Självkontroll}
\textit{När du använder din självkontroll för att stå emot psykisk påverkan eller påfrestning som stress, traumatiska upplevelser och övernaturliga krafter} slå \textbf{+Sinne}
\begin{itemize}
  \item[10+] Du biter ihop och kan fortsätta opåverkad.
  \item[7-9] Du står emot men ansträngningen ger dig ett tillstånd som varar tills du fått tillfälle att återhämta dig. Välj en:
  \begin{itemize}
    \item Du blir arg, ledsen, rädd eller skuldtyngd. (-1 Välmående).
    \item Du blir hänförd (+1 Relation till det som orsakar tillståndet).
    \item Du blir distraherad (-2 i situationer där tillståndet begränsar dig).
    \item Du hemsöks av upplevelsen senare (SL får en hållhake).
  \end{itemize}
  \item[2-6] Du förlorar kontrollen och SL väljer mellan att du är maktlös inför hotet, panikslagen utan kontroll över dina handlingar eller  mår dåligt av traumat (sänk Välmående enligt traumats styrka).
  \begin{itemize}
    \item Allvarligt trauma (-2 Välmående)
    \item Livsavgörande trauma (-4 Välmående)
  \end{itemize}
\end{itemize}
\section{Välmående}
Välmående är ett mått på hur pass balanserat rollpersonens psyke är. En rollperson är till en början kontrollerad men kan sjunka i Välmående när hon är med om traumatiska upplevelser.
\begin{table}[!h]
\begin{tabular}{|c| l l|}
\hline & Kontrollerad & \\
\hline & Olustig & \textbf{Lindrig stress:} \\
& Ofokuserad &  \textit{-1 Sinne} \\
\hline & Skakad &  \textbf{Allvarlig stress:} \\
& Stressad & \textit{-1 Självkontroll, -2 Sinne} \\
& Neurotisk &  \\
\hline & Ångestdrabbad &  \textbf{Kritisk stress:} \\
 & Irrationell &  \textit{-2 Självkontroll, -3 Sinne} \\
 & Okontrollerad &  \\
\hline & Nedbruten & Spelledaren gör ett drag \\
\hline
\end{tabular}
\end{table}
\section{Hjälpa eller motarbeta}
\textit{När du hjälper eller motarbetar en annan rollperson}, slå och lägg till ditt attribut för samma drag som den andra rollperson slår för.
\begin{itemize}
  \item[10+] Ge rollpersonen +2 eller -2 på slaget.
  \item[7-9] Ge rollpersonen +1 eller -1 på slaget.
  \item[2-6] Något går fel för dig. SL gör ett drag.
\end{itemize}


\clearpage
\part{Bakgrund}
\section{Timelocks}
Tidslås skapas genom att sätta upp enormt kraftfulla Timeflux enheter. Dessa enheter existerar i alla fyra dimensioner och är så kraftfulla att de inte går att förgöra förens deras kraftkällor börjar sina. Tidslåsen gör det näst intill omöjligt att resa tillbaka till den perioden. Ända sättet att manipulera tiden under ett tidslås är att helt enkelt bara ha en fungerande organisation som lever vanliga liv utan tidsresande.
\section{Agentaktiviteter under tidslåsen}
Det första stora tidskriget som pågått i 192 år avslutades 1502 när Leonardo Da Vinci offrade sitt liv för att etablera en enormt kraftull Timefluxenhet. Tidsresor in i tidskrigets slut blev väldigt frekventa och många justeringar genomfördes för att få in så många agenter som möjligt i olika hemliga sälskap. Illuminati, tempelherreorden och många andra slutna sällskap blev grogrunden för dessa agenter. De skapade skrifter om vad deras efterkommande skulle leta efter för tecken och hur de skulle bygga upp organisationen för att hålla människorna i skack.
\subsection{Rebellerna - Människorna}
Efter att Leonardo avslutat tidsresorna fanns inte många upplysta kvar vilket ledde till att enbart några få som avslöjat någon av agenternas riktiga agendor eller avhoppare från dessa känner till vad som egentligen "finns där ute". Förberedelserna inför tidslåsets upphörande blev därför få. Under 1980-talet är de mest tin foil hats. Och liknande nördar som ingen tror på.
\subsection{Cyprox incorperate - Förgörarna}
I det stora kriget föll Cyprox som de stora förlorarna då de förintats och deras sändpunkter i framtiden avslöjats för Ginnies. Några enstaka celler överlevde och enbart en mindre organisation lyckades hålla fast greppet genom tidens tand. Under 1980 talet finns de enbart kvar i nordkorea förklädda till den statens ledare.
\subsection{Ginnies - Avledarna}
Ginnies blev de stora vinnarna dels på grund av deras kamp mot Cyprox från den sändande epok där kriget mellan parterna också pågår. Ginnies slår sig i allians med Rebellerna och deras ledare Leonardo under slutet av 1480-talet och lyckas på så sätt till slut ta över. Efter att låset kommer på plats har deras tidsarkitekter kommit dit och dirigerar hårt vad som skall hända framåt. De sår sina frön utifrån den kunskap de besitter och lyckas väl med att etablera sig. De har fortfarande stor eller mycket stor maktposition under 1980 talet i form av olika celler runt om i världen. Deras största fästen är i Indien och i USA.
\part{1986 året för återkomsten}
\section{Marylain}
I framtiden sätts nya baser upp för tidsresor och probes skickas tillbaka med hundra års mellanrum, sedan 10 år och slutligen hittar man att vissa probes kommer igenom under början på 1986. Den första agenten kommer fram den 7'e februari. En liten ort i södra USA, Marylain kommer att bli dessa trevande försöks viktigaste plats. Då tidslåset fortfarande gör det mycket svårt att komma fram innan det är avstängt gör att enbart en handfull av resande kommer fram. Alla som når fram har olika uppdrag.
\section{Cyprox}
Cyprox är helt fel ute och tror att Ginnies sin vana trogen etablerar sig i Indien. De hamnar där och bygger sin cell i New Deli. Det tar dem ett antal år att hitta Ginnies agenter vilket gör att deras närvaro i USA är okänd under lång tid.
\section{Ginnies 1986}
Ginnies anländer i byn med två arkitekt en man, Tony Diego och en kvinna Ilena Garsia, och en hacker Jonathan Mercy. De två arkitekterna börjar genast genomföra de förändringar som skall leda fram till att tidslåset faller vid rätt tid. Tony tar anställning i skolan och påbörjar där att värva barn för att få bäst spridning på deras uppdrag. Utvalda elever \textit{vaccineras} och får the cricket gene inplaneterad i sig. De kommer att plockas ca 30 år senare för att vervas in som agenter. Ilena börjar istället på IRS kontoret och gifter sig snart för att få barn.
\subsection{Agenda}
Att börja tippa händelserna mot att tidslåset bryts 2018 så som de vet att det kan göra. För att nå dit behöver de se till att många nog från den lilla orten får möjligheten att utveckla en potent gen. Rätt personer skall sedan identifieras och rekryteras som agenter.
\subsection{Hotet}
\begin{itemize}
  \item[Låg] Arkitekterna börjar bygga upp sina epoker. De etablerar generna och planerar för vilka aktiviteter som behöver komma på plats.
  \item[1:a växeln] Mass spridning av generna genomförs. Utvalda klasser i skolan \textit{vaccineras} för att få genen. Epokerna börjar rullas ut.
  \item[2:a växeln] De vaccinerade börjar följas upp för att identifiera vilka det är. Tidiga tecken på vad som händer de påverkade boxas in.
  \item[3:e växeln] De identifierade familjerna infiltreras och barnen vervas som agenter i tidig ålder.
  \item[Overdrive] Blir de påkomna med allt för hårda bevis undanröjs bevismaterial och vitnen utan pardon.
\end{itemize}
\subsection{Roller}
\subsubsection{Tony Diego}
\textbf{Roll: Arkitekt}\\
\textbf{Drifkraft:} Att sätta upp förutsättningar för att tidslåset skall falla 2018. Tar anställning vid skolan och börjar påverka barnen för att de skall genomdriva hans agenda i framtiden.
\subsubsection{Ilena Garsia}
\bild{namn=Ilena-Garsia,sida=l}
\textbf{Roll: Arkitekt}\\
Drifkraft: Att sätta upp förutsättningar för att tidslåset skall falla 2018. Tar anställning vid sjukhuset och börjar påverka patienter och organisationen för att genomdriva agenda i framtiden.
\subsubsection{Jonathan Mercy}
% \bild{namn=Ginne-agent,sida=l}
\textbf{Roll: Hacker}\\
Drivkraft: Skaffa många barn som möjligt för att sprida en genen via blodet vilket ger bästa potensen. Tar anställning på IRS kontoret och börjar där bygga upp och koda om system för att förbereda för återkomsten.
\subsection{Platser}
\subsubsection{Marylain elementary high}
En traditionell highschool där de flesta av stadens ungdomar går. Skolan är byggd i två våningar med en stor ljusgård i mitten. Klassrummen ligger ut efter fasadens kanter och ljusgården är byggd i grå granit med balkonger runt om på andra våningen. Utanför huvudbyggnaden finns en asfalterad plan med basketplan och två större flyglar. Den ena flygeln inrymmer sjukstugan där \textit{Tony Diego} jobbar som Skolsyster.
\subsubsection{Saint Lucys hospital}
Saint Lucys hospital är ett litet sjukhus som ligger i hjärtat av Marylain. Sjukhuset har hög standard och kombinerar privat vård med den basala som finns för de som inte har råd med sjukförsäkringen. I sina mingröna och mintblå uniformer går personalen runt här. Ordningen är mycket god och patienterna är över lag väldigt nöjda med den vård som bedrivs. Sjukhusets ledning styrs av Ginnies och deras vänner. Här administreras och tas gene-sprutorna fram. Det hela görs som ett forskningsprojekt för ny diabetesmedicin, något som får stora medel varje år från privata givare. Genom att gå igenom bokföringen och kanske även genom att ha koll på hur forskningsområdet kring diabetes bedrivs finns en liten chans att man hittar att något fuffens är på gång.
\paragraph{Malinda Bell}
Receptionist som hjälper besökare till sjukhuset att hitta de patienter de skall besöka. Permanentat hår, bruna plastglasögon på halsrem. 50 årsåldern.
\paragraph{Paul Hampden}
Ung läkare med mörkt hår, är påväg till lunchrast, hjälper till med forskningen på övertid men är inte fullt insatt i läget. Känner väl till de olika faciliterna på sjukhuset.
\subsubsection{Lagerhus 13}
Ett av lagerhusen nere i hamnen har byggts om till kontor och hideout för Ginnie agenterna. Lagerhuset ligger i en del av hamnen som tidigare användes som fiskehamn men nu är övergiven. Denna plats används enbart i nödfall och har ett kassaskåp där hemliga dokument och planer finns lagrade. Dokumenten är skrivna på Silomitriksa och bör vara mycket svåra för spelarna att förstå ifall de skulle komma över dem då det språket börjar talas om 3000 år. Vissa ord kan kännas igen om någon är språkforskare på morderna indiska språk.
\subsubsection{IRS kontoret}
IRS sysslar med skattekontroller på både privatpersoner och företag. Här finns naturligt en massa register över befolkningen, hur de flyttar, vart de jobbar och så vidare. Det här är den perfekta platsen för våra arkitekter att övervaka och även planera och genomdriva sina sociala akrikteturer. På kontoret i Marylain jobbar Jonathan Mercy så snart han får chansen att ordna papper osv. Han kommer också att hjälpa andra som kommer resande med att etablera en trovärdig pappersexistens i den här tiden. Kontoret i Marylain är inte så stort, här arbetar ett 20 tal personer och det är ett litet kontor som är utlokaliserat och täcker in en större region än själva Marylain. Lokalen har traditionella kontor och är över lag ganska sunkigt med renovering senast på 60-talet.
\section{Rebeller 1986}
\subsection{Agenda}
I den lilla orten finns ett antal rednecks, en av dem, Ian Parker såg när Tony kom fram genom tidshoppet och är mycket mistänksam, han har också kontaktats av en avhoppare från Ginnie cellen på orten, en annan kille med alkoholproblem som heter Ben Mathews.
\subsection{Hotet}
\begin{itemize}
  \item[Normal] Rednecks do what rednecks does.
  \item[Låg] Då Ian berättar för Ben vad han sett så inser Ben vad som är på gång. Han börjar berätta om de uråldriga sägnerna om tidsresenärerna.
  \item[1:a växeln] Ian och Ben börjar försöka värva andra till sitt cause.
  \item[2:a växeln] Rebellerna har nu fått med ytterligare 3-4 stycken och de börjar sprida rykten om Tony.
  \item[3:e växeln] Rebellerna börjar hota Tony och söker även efter fler tidresenärer. Alla som beter sig konstigt är skyldiga.
  \item[Overdrive] Rebellerna har fått nog och nu skall Tony och de andra dö!
\end{itemize}
\subsection{Roller}
\subsubsection{Ian Parker}
\bild{namn=Rebel,sida=l}
Ian Parker såg när Tony kom fram genom tidshoppet och är mycket mistänksam, han har också kontaktats av en avhoppare från Ginnie cellen på orten, en annan kille med alkoholproblem som heter Ben Mathews.
\subsubsection{Ben Mathews}
\bild{namn=Rebel2,sida=r}
Avhoppare från Ginnies cell i Marylain. Alkoholproblem och desperat. Blir kompis med Ian Parker och inser att det är allvar när Ian berättar att han sett Francis tidshoppa in i nutiden.
\subsection{Platser}
\subsubsection{Stanlies bar and grill}
En klassisk amerikansk bar. Drivs av Tony Tomahawk, en indian som är väldigt råbarkad. Han är vän med Ian trotts att Ian kanske inte alltid har det mest högtravande språket när det kommer till Tony. Stället är kanske inte det mest välbesökta av hela byn, men det har en hel rad stammissar som kommer dit både då och då. Stället har ett gömt rum under ett av förrådsrummen där Tony samlat på sig vapen för en hel pluton rednecks. Tonyt ställer upp och delar ut vapen till behövande rednecks om så krävs för att försvara sig mot skumma tillresta.
\subsubsection{Marylain outback camping}
Maryland outback camping park är ett ställe som var tänkt att locka tursiter till träskområdet där de skulle kunna ställa upp sina fina trailers och ha barbeque partyn på kvällarna. Det slog aldrig igenom och är efter år utan underhåll väldigt slitet och sunkigt. Här står några gamla trailers uppställda som inte tar sig iväg från parken då de inte längre går att köra. Ian bor i en av dessa trailers. Dessutom finns här ett 10 tal andra boenden. Att bo här är högst osäkert då ingen riktigt bryr sig om folk som bor här ute. De kvinnor som sökt sig hit i despiration att få tak över huvudet har ofta fått påhälsning av sina ex-män och blivit både slagna och våldtagna.
\subsection{Bakgrundsscener 1986}
\subsubsection{Ginnies}
\satscen{
  typ = Spelledarscen,
  rubrik = När A's fick sitt agentuppdrag av sin faster innan hon försvann.,
  namn = A: ,
  plats = Hemma hos A's fasters coola funkisvilla.,
  birollett = A's Mamma,
  birolltva = A's Pappa,
  birolltre = A's Farbror
}
\satscen{
  typ = Spelledarscen,
  rubrik = När vi gjorde oss redo för skolresan till Frankrike,
  namn = B:,
  plats = I väntrummet utanför skolsysterns rum,
  birollett = B's Mamma,
  birolltva = Skolsyster,
  birolltre = B's bästa kompis Sara Jones!!!
}
\subsubsection{Rebellerna}
\satscen{
  typ = Spelledarscen,
  rubrik = När farsan sköt Old Tom,
  namn = C:,
  plats = I pappas trailer en regnig höstdag,
  birollett = Old Tom,
  birolltva = Cs pappa Ian Parker,
  birolltre = William Tombs Old Toms son,
}
\clearpage
\part{2018 året då tidslåset låses upp}
\section{Scenariots startpunkt}
Arkitekterna har nu förberett allt, det som behövs nu är några tillräckligt mäktiga människor som kan klara av att stänga ned tidslåset. Att göra det kräver en stark gennärvaro och resursfulla personer. Våra arkitekter har sett till att de utvalda är motiverade att komma tillbaka till Marylain för att starta igång med arbetet. De är inte initierade i varför men var och en kommer få ett frö i sin bakgrund som gör att triggern arkitekterna använder sig av görs extra verksam för dem.
\section{Scener}
På följande sätt är ledtrådarna i scenariot ihopkopplade.\\
\includegraphics[width=0.9\textwidth]{Cluemap}
\subsection{New Orleans, klassfesten}
Alla rollpersoner befinner sig förmodligen i New Orleans dit de flyttat efter att de blivit vuxna. Rollpersonerna kommer alla att delta i en reunionfest för de som bor i New Orleans eller de som har haft möjligheten att resa dit. En av rollpersonerna får vara den som har bjudit in till träffen och får beskriva vart festen hålls, hur den är planerad osv. Alla rollpersoner får beskriva varför de är där och hur de känner inför att återförenas med sina gamla klasskamrater från Marylain high. Då rollersonerna inte har träffats på typ 30 år så är detta ett utmärkt tillfälle att pressentera sig för varndra om vad man nu för tiden gör i livet mm.
\subsubsection{Namn}
Ella Martin\\
Tanner Billing\\
Marcella Fulmer\\
Richie Skern\\
Cathleen Palmer\\
Truman Freen\\
Alana Deacons\\
Jay Fane\\
Juliana Moland\\
Amado Fayneman\\
Tessa Fisher\\
Odis Evans\\
Rebekah Salford\\
Donn Steven\\
Sophia Fane\\
Tyler Papley
\subsubsection{NPCer med leads.}
\paragraph{Sara Jones}
Sara var klassens innetjej under mellanstadiet. Idag är hon hemmafru och bor fortfarande kvar i Marylain i villakvartert där. Hon har fyra barn och är gravid igen så hon dricker inget under festen och poängterar hela tiden hur duktig hennes man är och talar om sina barn och visar gärna bilder av dem. Hon är överviktig och även om hon är garvid så är det bara i andra månden. Hennes man Tony Tommahawk driver Stanlies bar and grill. Hon kommer dock att senare på kvällen visa sina misstankar mot både det ena och det andra. Hon kommer rikta misstankar som känns lite ogrundade mot skolan och mot sjukhuset.
\paragraph{Sandra Jefferson - filmen}
Sandra har plockat med sig en vhs spelare och har med sig en inspelning från klassresan i trean. På filmen spelar klassen baseball och några av föräldrarna är throwers och pitchers. Sandra är glad att få visa filmen som hennes pappa spelat in och den väcker kanske en del minnen från rollpersonerna som får beskriva själva vad de gör i filmen. I slutet av filmen så visas pappan som kastar, det är samma man som någon av rollpersonerna sett på vägen in till Marylain. Mannen har samma ålder som om inte något äldre än den man de såg. Ingen verkar veta vem det var som kastade, kanske kan de hitta vem det är i skolan?
\paragraph{William Tombs - dödssjuk}
Självmordsförsöket av William Tombs. Han har blivit mördad av Ginnies som vet att han kommer att försöka hindra rollpersonerna från att stänga tidslåset. Han är uppvuxen i Marylain camping park. Han har blivit förgiftad vid en vaccinering på sjukhuset och är i väldigt dåligt skick. Han kommer att skicka sms till någon av rollpersonerna om att han är sjuk och att han har blivit inlagd på sjukhuset och därför inte kan komma. Meddelandet kommer från Ginnie agenterna för att inte skapa någon form av misstanke.
\subsection{Saint Lucys hospital}
För miljöbeskrivning se sjukhuset i Ginnies platslista.\\
\bild{namn=ada,sida=l}
Sjukhuset har stora anslag om att de vill få forskningsresurer till sitt diabetesteam som jobbar med genombrytande forskning på just diabetesområdet. En doktor Ilena Garsia står som huvudforskare för prjektet på anslaget. Hon ler på bilden och ser inbjudande ut. Här finns en logotyp för ADA American Diabetes Association.
\subsubsection{William Tombs på sal 5}
William ligger på sal 5 på intensivvårdsavdelningen och är sovande. Han har krafter kvar för att bli väkt om narkosen stängs av. Han har information om att det sista han mins är att han brutit sig in på Lagerhus 13. Där skall det finnas information som pekar på en konspiration i Marylain. Han orkar inte tala så mycket mera innan han till slut somnar in för gott. Om rollpersonerna kollar på Williams tillhörigheter så kommer de att hitta hans hemadress, det är ute i \textit{Marylain outback camping}.
Ledtråd till Baren?
\subsection{Stanlies bar and grill}
\subsubsection{Larry Wade, mannen från filmen}
Här inne finns Larry Wade som kändes igen från filmen Sandra visade. Han kommer inte känna igen rollpersonerna. Om de visar honom filmen så kommer han att bli skakad men försöka bortförklara det med att det måste vara någon annan. Larry jobbar på sjukhuset som sjuksköterska. Han kan om rollpersonerna blir väldigt goda vänner med honom hjälpa dem in på sjukhuset för att undersöka ADA.
\subsubsection{Tony Tommahawk}
Tony kan bekräfta de misstankar som hans fru talat om på klassfesten. Han tror att både Skolan och IRS kontoret är korrupta och har varit under lång tid. Han och flera av hans stammissar har sett märkliga grejjer. X och Y står och tittar ut över havet och mäter solupgången med märkliga instrument.
Ledtråd till Skolan
Ledtråd till IRS
\subsection{Skolan}
Ledtråd till Baren
Ledtråd till Sjukhuset
Ledtråd till Lagerhus 13
\subsection{Lagerhus 13}
Information om att det finns en aparat som skyddas.
Ledtråd till Italien
Ledtråd till IRS
\subsection{IRS}
Ledtråd till Lagerhus 13
Ledtråd till Italien
\subsection{Mobile home}
Williams hem, här finns ett visitkort till ADA med en tid inskriven på baksidan.
\subsection{Italien}
\subsection{Tyskland}

\subsection{Den mystiska dagboken}

\subsection{Agenda}
agendan
\subsection{Hotet}
hotet
\subsection{Roller}
roll
\subsection{Platser}
plats
\subsection{Växlar}
\begin{itemize}
  \item[Normal] normala
  \item[Låg] låg
  \item[1:a växeln] 1
  \item[2:a växeln] 2
  \item[3:e växeln] 3
  \item[Overdrive] extremt
\end{itemize}

\section{Rebeller 1986}
vilka
\subsection{Agenda}
agendan
\subsection{Hotet}
hotet
\subsection{Roller}
roll
\subsection{Platser}
plats
\subsection{Växlar}
\begin{itemize}
  \item[Normal] normala
  \item[Låg] låg
  \item[1:a växeln] 1
  \item[2:a växeln] 2
  \item[3:e växeln] 3
  \item[Overdrive] extremt
\end{itemize}
\section{Ginnies 2018}
vilka
\subsection{Agenda}
agendan
\subsection{Hotet}
hotet
\subsection{Roller}
roll
\subsection{Platser}
plats
\subsection{Växlar}
\begin{itemize}
  \item[Normal] normala
  \item[Låg] låg
  \item[1:a växeln] 1
  \item[2:a växeln] 2
  \item[3:e växeln] 3
  \item[Overdrive] extremt
\end{itemize}

\section{Rebeller 2018}
vilka
\subsection{Agenda}
agendan
\subsection{Hotet}
hotet
\subsection{Roller}
roll
\subsection{Platser}
plats
\subsection{Växlar}
\begin{itemize}
  \item[Normal] normala
  \item[Låg] låg
  \item[1:a växeln] 1
  \item[2:a växeln] 2
  \item[3:e växeln] 3
  \item[Overdrive] extremt
\end{itemize}


\clearpage
\section{Gamla noteringar}

Ett karaktärspel där karaktärernas bakgrunder knyts in i scenariot. Spelstilen kommer vara starkt narativt driven och reglerna kommer baseras på Kult: Divinity lost men inte världen.

Det första scenariot kommer vara lite deckaraktigt där karaktärerna blir personligt indragna i mystiska försvinnanden.

Karaktärerna har i sin bakgrund varit med om försvinnanden i sin närhet men de har inte varit personliga utan de har helt enkelt bevittnat dessa, men när de var så pass unga att de inte blev trodda.

Ta fram en scen per karaktär där någon i deras omgivning försvinner spårlöst.

I början av scenariot så spelas en scen där någon närstående till karaktären försvinner spårlöst.



Agentscen där en agent rekryterar dem?

Kan agenten vara en av spelarna?

Uppdraget är att hjälpa dem reda ut vem som får folk att försvinna.

Hemliga uppdraget är att skydda dem från att bli tagna.

Hemresan till hemorten?

Platser att hitta med ledtrådar

Biblioteket/Skolbibblan
  Vad har hänt här egntligen? Vad hände när försvinnanden genomfördes? Vilken information skall leda till Ginnies?

Ginnie safe house
  Hur är det skyddat? Vilken verksamhet är dess front?

Timeflux maskinen
  Tungt bevakad
  Utan avstängningsfunktion

Turn off in all dimensions?
  Sight of the aliens?

Liten ort från början, alla har någon som försvinner. Flyttar därifrån på 90 talet, nu när det händer igen så ger ni er tillbaka.

Scen: Lektion i skolan, lärarinnan Asta berättar om Sveriges floder. Plötsligt försvinner hon mitt i en mening. Vad gör klassen.


\end{document}
