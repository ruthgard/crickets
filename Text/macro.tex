\usepackage{xkeyval}

\makeatletter
\define@cmdkey{rpg}[char@]{name}[]{}
\define@cmdkey{rpg}[char@]{class}[]{}
\define@cmdkey{rpg}[char@]{kropp}[]{}
\define@cmdkey{rpg}[char@]{reaktion}[]{}
\define@cmdkey{rpg}[char@]{sinne}[]{}
\define@cmdkey{rpg}[char@]{special}[]{}
\define@cmdkey{rpg}[char@]{width}[\linewidth]{}

\define@cmdkey{move}[char@]{rubrik}[]{}
\define@cmdkey{move}[char@]{beskrivning}[]{}
\define@cmdkey{move}[char@]{grundegenskap}[]{}
\define@cmdkey{move}[char@]{lyckat}[]{}
\define@cmdkey{move}[char@]{mitten}[]{}
\define@cmdkey{move}[char@]{misslyckat}[]{}
\define@cmdkey{move}[char@]{val}[]{}
\define@cmdkey{move}[char@]{width}[\linewidth]{}

\define@cmdkey{scenen}[char@]{typ}[]{}
\define@cmdkey{scenen}[char@]{namn}[]{}
\define@cmdkey{scenen}[char@]{rubrik}[]{}
\define@cmdkey{scenen}[char@]{plats}[]{}
\define@cmdkey{scenen}[char@]{birollett}[]{}
\define@cmdkey{scenen}[char@]{birolltva}[]{}
\define@cmdkey{scenen}[char@]{birolltre}[]{}
\define@cmdkey{scenen}[char@]{width}[\linewidth]{}

\newcommand{\statlabel}{\textsc}

\usepackage{calc,xparse}
\newlength\mycardwidth
\setlength\mycardwidth{.75\textwidth}
\NewDocumentCommand \boxme { O{} }{%
  \fbox{%
    \parbox{\linewidth-2\fboxrule-2.5\fboxsep}{\strut #1}%
  }%
}
\NewDocumentCommand \pageme { O{} O{} }{%
  \begin{minipage}{#1\mycardwidth}
    \centering\boxme[#2]
  \end{minipage}%
}
\NewDocumentCommand \cardme { O{.75\textwidth} +m }{%
  \setlength\mycardwidth{#1-2\fboxrule-2\fboxsep}%
  \fbox{%
    \begin{minipage}{\mycardwidth}
      \centering
      #2%
    \end{minipage}%
  }%
}

\newcommand{\setcharacterstats}[1]{%
  \begingroup
  \setkeys{rpg}{name,class,kropp,reaktion,sinne,special,width,#1}%
  \noindent
  \cardme[\textwidth]{%
    \pageme[.69][\statlabel{Namn}: \char@name]%
    \pageme[.31][\statlabel{\char@class}]

    \pageme[.33][\statlabel{Kropp}: \char@kropp]%
    \pageme[.34][\statlabel{Reaktion}: \char@reaktion]%
    \pageme[.33][\statlabel{Sinne}: \char@sinne]
    \raggedright \textit{Sätt ut +2, +1, 0}

    \pageme[][\statlabel{Specialdrag}:]
    \raggedright \textit{\char@special}
  }
}
\newcommand{\displaymove}[1]{%
  \begingroup
  \setkeys{move}{rubrik,beskrivning,grundegenskap,lyckat,mitten,misslyckat,val,width,#1}%
  \noindent
  \cardme[\textwidth]{%
    \pageme[.69][\statlabel{\char@rubrik}]%
    \pageme[.31][\statlabel{Slå} +\char@grundegenskap]

    \pageme[.99][\char@beskrivning]

    \pageme[.31][\statlabel{10+}]%
    \pageme[.69][\char@lyckat]

    \pageme[.31][\statlabel{7-9}]%
    \pageme[.69][\char@mitten]

    \pageme[.31][\statlabel{2-6}]%
    \pageme[.69][\char@misslyckat]

    \pageme[.99][\char@val]%
  }
}
\newcommand{\satscen}[1]{%
  \begingroup
  \setkeys{scenen}{typ,namn,rubrik,plats,birollett,birolltva,birolltre,width,#1}%
  \noindent
  \cardme[\textwidth]{%
    \pageme[1][\statlabel{\textbf{\char@typ}}]

    \pageme[1][\statlabel{Rubrik:} \char@rubrik]

    \pageme[1][\statlabel{Plats:} \char@plats]

    \pageme[1][\statlabel{Huvudroll:} \char@namn]

    \pageme[1][\statlabel{Biroller: } \char@birollett, \char@birolltva]

    \pageme[1][\statlabel{Biroller: } \char@birolltre]
  }
}
\newcommand{\rollformular}[1]{%
  \begingroup
  \setkeys{rpg}{name,class,kropp,reaktion,sinne,special,width,#1}%
  \noindent
  \cardme[\textwidth]{%
    \pageme[.69][\statlabel{Namn}: \char@name]%
    \pageme[.31][\statlabel{Koncept}:]
    \pageme[][\statlabel{Rädsla}:]
    \pageme[][\statlabel{Drivkraft}:]

    \pageme[.33][\statlabel{Kropp}: \char@kropp]%
    \pageme[.34][\statlabel{Reaktion}: \char@reaktion]%
    \pageme[.33][\statlabel{Sinne}: \char@sinne]
    \pageme[][\statlabel{Specialdrag}:]
  }
}

\define@cmdkey{bild}[bild@]{namn}[]{}
\define@cmdkey{bild}[bild@]{sida}[r]{}
\newcommand{\bild}[1]{%
  \begingroup
  \setkeys{bild}{namn,#1}%
  \begin{wrapfigure}{\bild@sida}{0.25\textwidth} %this figure will be at the right
      \centering
      \includegraphics[width=0.25\textwidth]{\bild@namn}
  \end{wrapfigure}
}
\makeatletter
