\documentclass[a5paper,10pt]{article}
\newcommand{\MyTitle}{De tillresta - Spelarbok}
\documentclass[a5paper,10pt]{article}

\usepackage[utf8]{inputenc}
\usepackage[swedish]{babel}
\usepackage[T1]{fontenc}
\usepackage[pdftex]{graphicx}
\usepackage{calc,xparse,geometry}
\usepackage{wrapfig}
\graphicspath{ {./../Bilder/} }
\author{Gustav Ruthgård}
\title{\MyTitle}
\date{\today}

\usepackage{xkeyval}

\makeatletter
\define@cmdkey{rpg}[char@]{name}[]{}
\define@cmdkey{rpg}[char@]{class}[]{}
\define@cmdkey{rpg}[char@]{kropp}[]{}
\define@cmdkey{rpg}[char@]{reaktion}[]{}
\define@cmdkey{rpg}[char@]{sinne}[]{}
\define@cmdkey{rpg}[char@]{special}[]{}
\define@cmdkey{rpg}[char@]{width}[\linewidth]{}

\define@cmdkey{move}[char@]{rubrik}[]{}
\define@cmdkey{move}[char@]{beskrivning}[]{}
\define@cmdkey{move}[char@]{grundegenskap}[]{}
\define@cmdkey{move}[char@]{lyckat}[]{}
\define@cmdkey{move}[char@]{mitten}[]{}
\define@cmdkey{move}[char@]{misslyckat}[]{}
\define@cmdkey{move}[char@]{val}[]{}
\define@cmdkey{move}[char@]{width}[\linewidth]{}

\define@cmdkey{scenen}[char@]{typ}[]{}
\define@cmdkey{scenen}[char@]{namn}[]{}
\define@cmdkey{scenen}[char@]{rubrik}[]{}
\define@cmdkey{scenen}[char@]{plats}[]{}
\define@cmdkey{scenen}[char@]{birollett}[]{}
\define@cmdkey{scenen}[char@]{birolltva}[]{}
\define@cmdkey{scenen}[char@]{birolltre}[]{}
\define@cmdkey{scenen}[char@]{width}[\linewidth]{}

\newcommand{\statlabel}{\textsc}

\usepackage{calc,xparse}
\newlength\mycardwidth
\setlength\mycardwidth{.75\textwidth}
\NewDocumentCommand \boxme { O{} }{%
  \fbox{%
    \parbox{\linewidth-2\fboxrule-2.5\fboxsep}{\strut #1}%
  }%
}
\NewDocumentCommand \pageme { O{} O{} }{%
  \begin{minipage}{#1\mycardwidth}
    \centering\boxme[#2]
  \end{minipage}%
}
\NewDocumentCommand \cardme { O{.75\textwidth} +m }{%
  \setlength\mycardwidth{#1-2\fboxrule-2\fboxsep}%
  \fbox{%
    \begin{minipage}{\mycardwidth}
      \centering
      #2%
    \end{minipage}%
  }%
}

\newcommand{\setcharacterstats}[1]{%
  \begingroup
  \setkeys{rpg}{name,class,kropp,reaktion,sinne,special,width,#1}%
  \noindent
  \cardme[\textwidth]{%
    \pageme[.69][\statlabel{Namn}: \char@name]%
    \pageme[.31][\statlabel{\char@class}]

    \pageme[.33][\statlabel{Kropp}: \char@kropp]%
    \pageme[.34][\statlabel{Reaktion}: \char@reaktion]%
    \pageme[.33][\statlabel{Sinne}: \char@sinne]
    \raggedright \textit{Sätt ut +2, +1, 0}

    \pageme[][\statlabel{Specialdrag}:]
    \raggedright \textit{\char@special}
  }
}
\newcommand{\displaymove}[1]{%
  \begingroup
  \setkeys{move}{rubrik,beskrivning,grundegenskap,lyckat,mitten,misslyckat,val,width,#1}%
  \noindent
  \cardme[\textwidth]{%
    \pageme[.69][\statlabel{\char@rubrik}]%
    \pageme[.31][\statlabel{Slå} +\char@grundegenskap]

    \pageme[.99][\char@beskrivning]

    \pageme[.31][\statlabel{10+}]%
    \pageme[.69][\char@lyckat]

    \pageme[.31][\statlabel{7-9}]%
    \pageme[.69][\char@mitten]

    \pageme[.31][\statlabel{2-6}]%
    \pageme[.69][\char@misslyckat]

    \pageme[.99][\char@val]%
  }
}
\newcommand{\satscen}[1]{%
  \begingroup
  \setkeys{scenen}{typ,namn,rubrik,plats,birollett,birolltva,birolltre,width,#1}%
  \noindent
  \cardme[\textwidth]{%
    \pageme[1][\statlabel{\textbf{\char@typ}}]

    \pageme[1][\statlabel{Rubrik:} \char@rubrik]

    \pageme[1][\statlabel{Plats:} \char@plats]

    \pageme[1][\statlabel{Huvudroll:} \char@namn]

    \pageme[1][\statlabel{Biroller: } \char@birollett, \char@birolltva]

    \pageme[1][\statlabel{Biroller: } \char@birolltre]
  }
}
\newcommand{\rollformular}[1]{%
  \begingroup
  \setkeys{rpg}{name,class,kropp,reaktion,sinne,special,width,#1}%
  \noindent
  \cardme[\textwidth]{%
    \pageme[.69][\statlabel{Namn}: \char@name]%
    \pageme[.31][\statlabel{Koncept}:]
    \pageme[][\statlabel{Rädsla}:]
    \pageme[][\statlabel{Drivkraft}:]

    \pageme[.33][\statlabel{Kropp}: \char@kropp]%
    \pageme[.34][\statlabel{Reaktion}: \char@reaktion]%
    \pageme[.33][\statlabel{Sinne}: \char@sinne]
    \pageme[][\statlabel{Specialdrag}:]
  }
}

\define@cmdkey{bild}[bild@]{namn}[]{}
\define@cmdkey{bild}[bild@]{sida}[r]{}
\newcommand{\bild}[1]{%
  \begingroup
  \setkeys{bild}{namn,#1}%
  \begin{wrapfigure}{\bild@sida}{0.25\textwidth} %this figure will be at the right
      \centering
      \includegraphics[width=0.25\textwidth]{\bild@namn}
  \end{wrapfigure}
}
\makeatletter

\begin{document}
\maketitle
\clearpage
\section{Introduktion}
Det här rollspelets regler baseras på idéer ur Apocalypse World och kan därför räknas som ett så kallat PbtA spel, eller Powered by the Apocalypse spel. Spelardragen är inspirerade av Kult: Divinity Losts spelardrag.

Alla spelare kommer ha karaktärer som är födda i den lilla orten Marylain som ligger 6 mil väster om New Orleans i den amerikanska södern. Din karaktär är född 1977 och har någon gång mellan 1995 och 1999 flyttat ifrån Marylain för jobb eller högre studier.
\section{Koncept}
Ditt koncept är grundinriktningen för din rollperson, det definierar vilken typ av karaktär den du spelar har, det ger också en känsla för vilken bakgrund karaktären har. Följande koncept finns att välja bland:
\begin{itemize}
  \item Agenten
  \item Korren
  \item Atleten
  \item Familjevårdaren
  \item Politikern
  \item Förbrytaren
\end{itemize}
\subsection{Grundegenskaper}
I det här rollspelet finns tre grundegenskaper. När du slår för ett spelardrag använder du två sexsidiga tärningar och lägger ihop resultatet med grundegenskapen.
\begin{itemize}
  \item \statlabel{Kropp} beskriver din fysiska förmåga i spelet. Både styrka och smidighet ryms inom begreppet kropp. Följande drag baseras på kropp.
  \begin{itemize}
    \item Din förmåga att attackera.
    \item Din förmåga att undvika fysiska utfall
    \item Din förmåga att uthärda fysisk skada.
  \end{itemize}
  \item \statlabel{Reaktion} beskriver din förmåga att uppfatta vad som händer runt din rollperson. Följande drag baseras på reaktion.
  \begin{itemize}
    \item Din förmåga att överblicka en situation.
    \item Din förmåga att läsa av en person.
  \end{itemize}
  \item \statlabel{Sinne} beskriver hur vässat din rollpersons sinne är. Följande drag baseras på sinne.
  \begin{itemize}
    \item Din förmåga att efterforska information eller att undersöka en plats.
    \item Din förmåga att övertala andra på olika sätt.
    \item Din förmåga att behålla kontrollen.
  \end{itemize}
\end{itemize}
\subsection{Rädsla och drivkraft}
Din rädsla och din drivkraft definierar din karaktär på ett viktigt sätt och hjälper dig att ta beslut utifrån din karaktärs perspektiv istället för ditt eget.
Du väljer ett av varje i listan på det koncept du valt eller väljer ett eget i samråd med spelledaren.
\subsection{Att bli bättre}
Varje gång du misslyckas helt med ett drag och får 6 eller lägre så noterar du en erfarenhetspoäng, man lär sig av sina misstag. Efter varje spelmöte delas också erfarenhetspoäng enligt listan:
\begin{itemize}
  \item Har du lärt dig något nytt om världen? Ta 1 erfarenhetspoäng.
  \item Har du ensam eller tillsammans med gruppen löst ett större problem? Ta 2 erfarenhetspoäng.
  \item Har någon annan gjort något som bidragit till ett bra spelmöte? Ge en erfarenhetspoäng till den.
\end{itemize}
\clearpage
\part{Koncept}
\section{Agenten}
\textit{Du arbetar som någon form av rättvisans beskyddare, det kan till exempel vara en polis, utredare, detektiv, eller hemlig polis. Att ordningen i samhället upprätthålls är ditt ansvar. Jobbet är på obekväma tider och du förväntas kunna ställa upp när som helst.}
\vskip1em
\setcharacterstats{
  class = Agent,
  special = Välj 1 av Skytt Detektiv eller Förhörsteknik
  }
\subsection{Specialdrag}
Välj ett specialdrag för din karaktär.
\subsubsection{Skytt}
\textit{När du agerar i strid med ett skjutvapen}, agerar du med +2 i modifikation.
\subsubsection{Detektiv}
\textit{När du undersöker något brottsrelaterat}, agerar du med +2 i modifikation.
\subsubsection{Förhörsteknik}
\textit{När du förhör någon}, agerar du med +2 i modifikation.
\subsection{Rädsla}
Välj en rädsla eller hitta på en egen till din karaktär.
\begin{itemize}
  \item Att förlora jobbet.
  \item Att inte kunna skydda dina vänner.
  \item Att något ur din bakgrund blir uppdagat av dina vänner.
\end{itemize}
\subsection{Drivkraft}
Välj en drivkraft eller hitta på en egen till din karaktär.
\begin{itemize}
  \item Lagen dömmer alla lika.
  \item Med lagen som förtecken finner jag makt.
  \item Jag är här för att hjälpa de svaga.
\end{itemize}
\clearpage
\section{Korren}
\textit{Du arbetar med någon form av journalistiskt yrke, det kan till exempel vara en tidningsreporter, redaktör, korrespondent, vloggare eller grävande journalist. Att allmänheten får reda på vad som händer i samhället är ditt ansvar.}
\vskip1em
\setcharacterstats{
  class = Korre,
  special = Välj 1 av Informationssökning Intervjuvteknik eller Press
  }
\subsection{Specialdrag}
Välj ett specialdrag för din karaktär.
\subsubsection{Informationssökning}
\textit{När du söker information}, agerar du med +2 i modifikation.
\subsubsection{Intervjuvteknik}
\textit{När du frågar ut någon}, agerar du med +2 i modifikation.
\subsubsection{Press}
\textit{När du agerar under hot på allmän plats}, agerar du med +2 i modifikation om du kan yrka på att du är Press.
\subsection{Rädsla}
Välj en rädsla eller hitta på en egen till din karaktär.
\begin{itemize}
  \item Att bli skadad av de som hotar dig.
  \item Att ditt material inte skall publiceras.
  \item Att inte ha råd att betala hyran.
\end{itemize}
\subsection{Drivkraft}
Välj en drivkraft eller hitta på en egen till din karaktär.
\begin{itemize}
  \item Sanningen måste fram.
  \item Allas påståenden måste ifrågasättas.
  \item Det finns mer som döljer sig i världen än vad folk tror.
\end{itemize}
\clearpage
\section{Atleten}
\textit{Du arbetar med någon form av atletiskt yrke, det kan till exempel vara en elitidrottare, gymnastiklärare, eller fysioterapeft. Att hålla din kropp i sjukt god kondition och även att motivera andra till det samma är din vardag.}
\vskip1em
\setcharacterstats{
  class = Atlet,
  special = Välj 1 av Närstrid Taktiker eller Lagspelare
  }
\subsection{Specialdrag}
Välj ett specialdrag för din karaktär.
\subsubsection{Närstrid}
\textit{När du agerar i en strid och din kroppsliga fysik är avgörande}, agerar du med +2 i modifikation.
\subsubsection{Taktiker}
\textit{När du agerar i en situation där praktisk taktik avgör utgången}, agerar du med +2 i modifikation.
\subsubsection{Lagspelare}
\textit{När du hjälper eller hindrar någon annan att agera i en farlig eller taktisk situation}, agerar du med +2 i modifikation.
\subsection{Rädsla}
Välj en rädsla eller hitta på en egen till din karaktär.
\begin{itemize}
  \item Att bli så skadad att du inte kan fortsätta tävla.
  \item Att bli utnyttjad igen.
  \item Att skada någon annan på riktigt.
\end{itemize}
\subsection{Drivkraft}
Välj en drivkraft eller hitta på en egen till din karaktär.
\begin{itemize}
  \item Att ständigt utmana din egen förmåga.
  \item Att stå i centrum och få allas beundran.
  \item Att visa för din far/mor att du duger.
\end{itemize}
\clearpage
\section{Familjevårdaren}
\textit{Du är den som tar hand om hemmet, familjen och din partner. Du är hemmaman eller hemmafru och är extremt bra på det. Att familjen fungerar väl, har en fungerande social kontakt med de andra i vänkretsen och att hemmet är i toppskick med skinande ytor och toppiffade fönsterbrädor är din förtjänst.}
\vskip1em
\setcharacterstats{
  class = Familjevårdare,
  special = Välj 1 av Minglaren Silvertunga eller Behärskad
  }
\subsection{Specialdrag}
Välj ett specialdrag för din karaktär.
\subsubsection{Minglaren}
\textit{När du småpratar med någon så får du alltid reda på något intressant}, agerar du med +2 i modifikation för att läsa av en person.
\subsubsection{Silvertunga}
\textit{När du försöker få din vilja igenom}, agerar du med +2 i modifikation när du försöker övertala en person.
\subsubsection{Behärskad}
\textit{Du har en mycket fast världsbild och upprätthåller den till varje pris}, du agerar du med +2 i modifikation när din självkontroll sätts på prov.
\subsection{Rädsla}
Välj en rädsla eller hitta på en egen till din karaktär.
\begin{itemize}
  \item Att skilsmässan är ett faktum.
  \item Att tappa ansiktet i din sociala umgängeskrets.
  \item Att någon i familjen far illa.
\end{itemize}
\subsection{Drivkraft}
Välj en drivkraft eller hitta på en egen till din karaktär.
\begin{itemize}
  \item Att ha den perfekta tillvaron.
  \item Att stå i centrum och få allas beundran.
  \item Att dina barn lyckas i livet.
\end{itemize}
\clearpage
\section{Politikern}
\textit{Du arbetar med någon form av politiskt yrke, det kan till exempel vara en lokalpolitiker, affärsman, eller facklig företrädare. Att driva igenom din ideologi och skaffa så många anhängare som möjligt är din passion. Att skapa allianser med de som står nära din ideologi och se dina meningsmotståndare på fall är din metod.}
\vskip1em
\setcharacterstats{
  class = Politiker,
  special = Välj 1 av Debattören Hårdhudad och Påläst
  }
\subsection{Specialdrag}
Välj ett specialdrag för din karaktär.
\subsubsection{Debattör}
\textit{När du för en argumentation med någon och vill få dem över på din sida}, agerar du med +2 i modifikation i övertala.
\subsubsection{Hårdhudad}
\textit{När du försöker få din vilja igenom}, agerar du med +2 i modifikation när du försöker övertala en person.
\subsubsection{Påläst}
\textit{Du har en mycket fast världsbild och upprätthåller den till varje pris}, du agerar du med +2 i modifikation när din självkontroll sätts på prov.
\subsection{Rädsla}
Välj en rädsla eller hitta på en egen till din karaktär.
\begin{itemize}
  \item Att ditt misstag kommer fram och din karriär är över.
  \item Att inte komma med i nästa val.
  \item Att vara utblottad när din karriär är över.
\end{itemize}
\subsection{Drivkraft}
Välj en drivkraft eller hitta på en egen till din karaktär.
\begin{itemize}
  \item Jag är politiker och därför rätteligen privilegierad.
  \item Min ideologi är den rätta.
  \item Jag är politiker för att skydda de svaga i samhället.
\end{itemize}
\clearpage
\section{Förbrytaren}
\textit{Du är någon form av förbrytare, det kan till exempel vara en småtjuv, gängmedlen, konsttjuv eller lurendrejare. Vad som är moraliskt eller juridiskt okej är inte regler som rör dig, du har andra regler att följa i den sociala grupp du befinner dig i. Det finns en anledning till att du är där du är idag, det är omständigheter som har lett fram till den situation du har idag. Det är inte ditt fel...}
\vskip1em
\setcharacterstats{
  class = Förbrytare,
  special = Välj 1 av Hänsynslös Skrämmande Ta sig in eller Skitliv
  }
\subsection{Specialdrag}
Välj ett specialdrag för din karaktär.
\subsubsection{Hänsynslös}
\textit{När du agerar i strid gör du det utan hämningar}, du agerar med +2 i modifikation.
\subsubsection{Ta sig in}
\textit{När du försöker ta dig förbi ett lås med dina universalverktyg} slå +Smidighet.
\begin{itemize}
  \item[10+] Låset öppnas snabbt och tyst.
  \item[7-9] Låset öppnas men, välj:
  \begin{itemize}
    \item det tar längre tid
    \item det gör extra oväsen
  \end{itemize}
  \item[2-6] Låset går upp men det tar tid och gör oväsen och välj:
  \begin{itemize}
    \item de verktyg du använder går sönder och kan inte användas igen
    \item det blir stor åverkan på låset.
  \end{itemize}
\end{itemize}
\subsubsection{Skrämmande}
\textit{När du förhör någon på ett hotfullt sätt}, agerar du med +2 i modifikation.
\subsubsection{Skitliv}
\textit{Du är van vid att dåliga saker händer dig, när du utsätts för något traumatiskt eller stressande}, agerar du med +2 i modifikation.
\subsection{Rädsla}
Välj en rädsla eller hitta på en egen till din karaktär.
\begin{itemize}
  \item Att bli dödad av mina egna.
  \item Att min familj får reda på vad jag sysslar med.
  \item Att bli ensam.
  \item Att inte få nästa fix.
\end{itemize}
\subsection{Drivkraft}
Välj en drivkraft eller hitta på en egen till din karaktär.
\begin{itemize}
  \item Ingen bestämmer över mig.
  \item Mina fiender är rädda för mig.
  \item Att göra en sista stöt för att bli fri.
\end{itemize}
\clearpage
\part{Bakgrundsscener}
Skapa fyra scener enligt reglerna i Bakgrundsskapande. En av scenerna du skapar kommer vara hemlig och byggas ut av dig och spelledaren tillsammans. Du kommer få ytterligare en scen av spelledaren som blir känd för alla spelare.

\satscen{
  typ = Hemlig scen
}
\satscen{
  typ = Spelledar scen
}
\satscen{
  typ = Öppen scen
}
\satscen{
  typ = Öppen scen
}
\satscen{
  typ = Öppen scen
}
\clearpage
\part{Spelardrag}
Din rollperson kan utföra alla dessa drag som en följd av att du agerar på olika sätt i en scen. När du säger vad din rollperson gör kan SL då be dig slå för ett av dessa drag.
\section{Attackera}
\textit{När du attackerar någon som kämpar tillbaka välj hur}

och slår \textbf{+Kropp}
\begin{itemize}
  \item[10+] Du lyckas och undviker motståndarens attack.
  \item[7-9] Du attackerar motståndaren men till ett pris. Spelledaren väljer 1:
  \begin{itemize}
    \item Du träffas av ett motanfall.
    \item Du gör mindre skada.
    \item Du förlorar något.
    \item Du gör av med all ammo.
    \item Du utsätts för ett nytt hot.
    \item Du får senare problem.
  \end{itemize}
  \item[2-6] SL gör ett mjukt eller hårt drag.
\end{itemize}
\clearpage
\section{Undvika}
\textit{När du undviker, blockerar eller parerar skada} slå \textbf{+Kropp}.
\begin{itemize}
  \item[10+] Du klarar dig helt oskadd.
  \item[7-9] Du klarar dig undan det värsta av skadan men spelledaren väljer om du hamnar i ett dåligt läge, förlorar något eller om du inte helt undviker skada.
  \item[2-6] Du reagerar för långsamt eller gör en felbedömning: kanske undviker du inte alls eller så hamnar du i ett värre läge än du började i. SL gör ett drag.
\end{itemize}
\section{Tar skada}
\textit{När du utsätts för skada} slå \textbf{+Kropp}. Har du skydd lägger du till värdet till slaget.
\begin{itemize}
  \item[10+] Du biter ihop och kan fortsätta som vanligt.
  \item[7-9] Du står fortfarande på benen men spelledaren väljer:
  \begin{itemize}
    \item Skadan får dig att hamna ur balans
    \item Du tappar något
    \item Du får ett \textbf{allvarligt sår}
  \end{itemize}
  \item[2-6] Skadan är överväldigande. Du väljer om du:
  \begin{itemize}
    \item Är utslagen för resten av scenen(SL avgör om du även får ett \textbf{allvarligt sår}).
    \item Får ett \textbf{kritisk sår} men är vid medvetande (har du redan ett kritiskt sår kan du inte välja detta alternativ igen).
    \item Dör.
  \end{itemize}
\end{itemize}
\subsection{Allvarligt sår}
Såret behöver någon typ av vård eller tid för att läka men blir inte värre av sig självt. Alkohol och smärtstillande droger kan ta bort avdraget som såret ger om så bara tillfälligt.
\subsection{Kritiskt sår}
Såret kommer inte läka av sig själv utan förvärras. Den som är kritisk sårad måste ha vård inom kort för att inte dö.
\subsection{Avdrag från sår}
Om du har icke-stabiliserade allvarliga och/eller kritiska sår drabbas du av avdrag, enligt nedan.
Om du har
\begin{itemize}
  \item ...minst ett allvarligt sår; -1 Kropp
  \item ...ett kritiskt sår; -1 Kropp
  \item ...både minst ett allvarlig sår och ett kritiskt sår; -2 Kropp
\end{itemize}
Om du dricker alkohol, tar smärtstillande, eller bedövar din smärta på liknande sätt, neutraliserar du avdraget från dina \textbf{allvarliga sår} under en kortare tidsperiod, vanligen under en scen. Detta gäller inte kritiska sår.
\clearpage
\section{Överblicka}
\textit{När du överblickar situationen} slå \textbf{+Reaktion}. Vid lyckat kan du ställa frågor till spelledaren. När du agerar på spelledarens råd ta +1 på ditt slag.
\begin{itemize}
  \item[10+] Ställ 2 frågor.
  \item[7-9] Ställ 1 fråga.
  \item[2-6] Du får ställa en fråga ändå men du drar till dig oönskad uppmärksamhet eller utsätter dig för fara.
\end{itemize}
Frågor
\begin{itemize}
  \item Vad är min bästa väg förbi hindret?
  \item Vad är det största hotet mot mig?
  \item Vad kan jag använda till min fördel?
  \item Vad behöver jag vara vaksam på?
  \item Finns det något gömt här?
  \item Är det något som är underligt?
\end{itemize}
\section{Läsa av en person}
\textit{När du läser av en person} slå \textbf{+Reaktion}
\begin{itemize}
  \item[10+] Ställ 2 frågor.
  \item[7-9] Ställ 1 fråga.
  \item[2-6] Du får ställa en fråga ändå men du drar till dig oönskad uppmärksamhet eller utsätter dig för fara.
\end{itemize}
Medan du talar med personen du läser av kan du spendera dina frågor 1 för 1 för att ställa deras spelare/spelledaren frågor:
\begin{itemize}
  \item Ljuger du?
  \item Vad känner du just nu?
  \item Vad tänker du göra?
  \item Vad önskar du att jag gör?
  \item Hur kan jag få dig att ... ?
  \item Är det något som är underligt?
\end{itemize}
\section{Undersöka}
\textit{När du undersöker någonting}, slå \textbf{+Sinne}. Om du lyckas så finner du alla direkta ledtrådar och får ställa frågor för att få mera information.
\begin{itemize}
  \item[10+] Fråga två frågor
  \item[7-9] Fråga en fråga, men svaret kostar dig något. SL bestämmer vad; du behöver någon eller något för att förstå svaret, eller det tar dig extra tid att få reda på svaret.
  \item[2-6] Du får fråga en fråga ändå, men du utsätts för oväntad fara eller annan kostnad.
\end{itemize}
Frågor:
\begin{itemize}
 \item Hur kan jag få reda på mer om vad jag undersöker?
 \item Vad säger min magkänsla om vad jag undersöker?
 \item Är det något konstigt med det jag undersöker?
\end{itemize}
\section{Övertala}
\textit{När du övertalar en spelledarperson genom förhandling, argumentation eller utifrån en maktposition} slå \textbf{+Sinne}.
\begin{itemize}
  \item[10+] Hon ger vika.
  \item[7-9] Hon gör som du vill men (SL väljer):
  \begin{itemize}
    \item Hon är inte nöjd och ber om mer i gengäld.
    \item Övertalningen skapar problem vid ett senare tillfälle.
    \item Hon ger med sig men är osäker (övertalningens effekt är bara tillfällig).
  \end{itemize}
  \item[2-6] Övertalningsförsöket har oavsiktliga konsekvenser. SL gör ett drag.
\end{itemize}

\textit{När du försöker övertala en rollperson} slå \textbf{+Sinne}.
\begin{itemize}
  \item[10+] Rollpersonen väljer själv om hon ger med sig eller inte, du ger dock båda effekterna nedan.
  \item[7-9] Rollpersonen väljer själv om hon ger med sig eller inte, du väljer dock en effekt nedan.
  \item[2-6] Rollpersonen är fri att göra som hon vill och har +1 på nästa slag mot dig. Ingen av effekterna nedan gäller.
\end{itemize}
Effekter:
\begin{itemize}
  \item Hon motiveras att göra som du vill (hon får +1 på nästa slag).
  \item Hon drabbas av tvivel om hon inte gör som du vill (hon får -1 Välmående).
\end{itemize}
\section{Agera under hot}
\textit{När du gör något riskabelt, är under tidspress eller försöker undkomma fara, berättar SL vad hotet är}, slå \textbf{+Sinne} för att agera trotts hotet.
\begin{itemize}
  \item[10+] Du gör det.
  \item[7-9] Du gör det, men tvekar, blir fördröjd, eller måste reagera på en komplikation - SL ger dig ett oväntat resultat, ett högt pris eller ett svårt val.
  \item[2-6] Det blir konsekvenser, du gör misstag eller så utsätts du för fara. SL gör ett mjukt eller hårt drag.
\end{itemize}
\section{Självkontroll}
\textit{När du använder din självkontroll för att stå emot psykisk påverkan eller påfrestning som stress, traumatiska upplevelser och övernaturliga krafter} slå \textbf{+Sinne}
\begin{itemize}
  \item[10+] Du biter ihop och kan fortsätta opåverkad.
  \item[7-9] Du står emot men ansträngningen ger dig ett tillstånd som varar tills du fått tillfälle att återhämta dig. Välj en:
  \begin{itemize}
    \item Du blir arg, ledsen, rädd eller skuldtyngd. (-1 Välmående).
    \item Du blir hänförd (+1 Relation till det som orsakar tillståndet).
    \item Du blir distraherad (-2 i situationer där tillståndet begränsar dig).
    \item Du hemsöks av upplevelsen senare (SL får en hållhake).
  \end{itemize}
  \item[2-6] Du förlorar kontrollen och SL väljer mellan att du är maktlös inför hotet, panikslagen utan kontroll över dina handlingar eller  mår dåligt av traumat (sänk Välmående enligt traumats styrka).
  \begin{itemize}
    \item Allvarligt trauma (-2 Välmående)
    \item Livsavgörande trauma (-4 Välmående)
  \end{itemize}
\end{itemize}
\section{Välmående}
Välmående är ett mått på hur pass balanserat rollpersonens psyke är. En rollperson är till en början kontrollerad men kan sjunka i Välmående när hon är med om traumatiska upplevelser.
\begin{table}[!h]
\begin{tabular}{|c| l l|}
\hline & Kontrollerad & \\
\hline & Olustig & \textbf{Lindrig stress:} \\
& Ofokuserad &  \textit{-1 Sinne} \\
\hline & Skakad &  \textbf{Allvarlig stress:} \\
& Stressad & \textit{-1 Självkontroll, -2 Sinne} \\
& Neurotisk &  \\
\hline & Ångestdrabbad &  \textbf{Kritisk stress:} \\
 & Irrationell &  \textit{-2 Självkontroll, -3 Sinne} \\
 & Okontrollerad &  \\
\hline & Nedbruten & Spelledaren gör ett drag \\
\hline
\end{tabular}
\end{table}
\section{Hjälpa eller motarbeta}
\textit{När du hjälper eller motarbetar en annan rollperson}, slå och lägg till ditt attribut för samma drag som den andra rollperson slår för.
\begin{itemize}
  \item[10+] Ge rollpersonen +2 eller -2 på slaget.
  \item[7-9] Ge rollpersonen +1 eller -1 på slaget.
  \item[2-6] Något går fel för dig. SL gör ett drag.
\end{itemize}

\clearpage
\includepdf[pages=-]{Tillresta-formular.pdf}
\end{document}
